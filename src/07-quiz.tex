\documentclass[a4paper,ngerman,12pt]{exam}
\usepackage{babel}
\usepackage[utf8]{inputenc}
\usepackage[T1]{fontenc}
\usepackage{graphicx}
\usepackage{algpseudocode}
\usepackage{geometry}
\usepackage{csquotes} % Anführungszeichen
\usepackage{paralist} % kompakte Aufzählungen
\usepackage{textcomp,tikz} %diverses
\usepackage{amsmath,amssymb,amstext,amsthm}
\usepackage{listings}
\usepackage{mathtools}
\usepackage{mdframed} % Boxen
\usepackage{float}
\usepackage{tikz}
\usetikzlibrary{calc}
\usetikzlibrary{arrows, automata}

\geometry{a4paper, top=3cm, left=2.7cm, right=2.7cm}
\pagestyle{plain}
\renewcommand{\solutiontitle}{\noindent\textbf{Lösung:}\enspace}
\DeclarePairedDelimiter\ceil{\lceil}{\rceil}
\DeclarePairedDelimiter\floor{\lfloor}{\rfloor}
\renewcommand\L{\mathcal{L}}

%\printanswers

\begin{document}
\noindent Theoretische Informatik \hfill Gruppe 8 \\
\mbox{}\hfill Loris Reiff
\begin{center}
  \bfseries\Large
  Quiz 7\ifprintanswers
  -- Lösungen
\fi
\end{center}

\begin{questions}
  \question
  Sei $\Sigma$ ein Alphabet und $L \subseteq \Sigma^*$, dann
  \begin{checkboxes}
    \CorrectChoice $L \in \L_{\mathrm{RE}} \wedge
      L^\texttt{C} \in \L_{\mathrm{RE}} \implies L \in \L_{\mathrm{R}}$
    \CorrectChoice $L \in \L_{\mathrm{RE}} \wedge
      L^\texttt{C} \in \L_{\mathrm{RE}} \impliedby L \in \L_{\mathrm{R}}$
    \CorrectChoice $L \in \L_{\mathrm{RE}} \wedge
      L^\texttt{C} \in \L_{\mathrm{RE}} \iff L \in \L_{\mathrm{R}}$
    \choice Keine der Aussagen ist korrekt
  \end{checkboxes}
\begin{solution}
Informal justification for
    $L \in \L_{RE} \wedge L^C \in \L_{RE} \Rightarrow L \in \L_R$:

    If $L \in \L_{RE} \wedge L^C \in \L_{RE}$ we know that there
    exists a TM $M$ such that $L = L(M)$ and a TM $M'$ with $L^C = L(M')$.
    We can therefore create a TM $\overline{M}$ that alternately executes a state
    in $M$ and $M'$ for an input $x$.
    We note that $q_{accept}$ and $q_{reject}$ of $M$ become
    $q_{accept}$ and $q_{reject}$ of $\overline{M}$.
    And $q_{accept}$ of $M'$ becomes $q_{reject}$ of $\overline{M}$.
    $q_{reject}$ of $M'$ becomes $q_{accept}$ of $\overline{M}$.

    If $x$ is in $L$, then $\overline{M}$ is guaranteed to terminate in
    $q_{accept}$ because of $M$. If $x$ is not in $L$, $\overline{M}$ is
    definitely going to halt in $q_{reject}$ because of $M'$.

    $\Leftarrow$ follows from Lemme 5.4 and $\L_{\mathrm{R}} \subsetneq \L_{\mathrm{RE}}$
\end{solution}

\question Sei
    $L = \{\text{Kod}(M) \mid M \text{ ist eine TM die Primzahlen akzeptiert}\}$
  dann gilt

  \begin{checkboxes}
    \choice $L \in \L_\mathrm{R}$
    \CorrectChoice $L \not\in \L_\mathrm{R}$
  \end{checkboxes}

  Begründe:
\begin{solutionorbox}[10em]
  $L$ ist die Sprache der Turingmaschinen, welche $L_P = \{p \mid p \text{ ist Prim}\}$
  akzeptiert. Es gilt $L \neq \emptyset$, ausserdem $L_P$ ist offensichtlich rekursiv.
  Somit ist $L$ ein semantisch nichttriviales Entscheidungsproblem über Turingmaschinen und
  gemäss Satz von Rice $L \not\in \L_\mathrm{R}$
\end{solutionorbox}

\question Sei
    $L = \{\text{Kod}(M) \mid M \text{ hält nie}\}$
    \begin{parts}
      \part Bestimme $L^\texttt{C}$
      \uplevel{\begin{solutionorbox}[4em] $ $ \\
    \vspace*{-2em}
  \begin{align*}
    L^\texttt{C} = &\{w \in \left(\Sigma_{\text{bool}}\right)^* \mid w \neq
  \mbox{Kod}(M) \text{ für alle TM }M\} \\
    &\cup \{\mbox{Kod}(M) \mid \exists x \text{ so dass TM } M \text{ auf } x \text{ hält}\}
  \end{align*}
\end{solutionorbox}}
      \part Zeige $L^\texttt{C} \in \L_{\text{RE}}$
    \uplevel{\begin{solutionorbox}[14em]
      Wir beschreiben eine NTM $M$, so dass $L(M) = L$.
      Für jedes Wort $w\in L^\texttt{C}$ gibt es eine endliche akzeptierende Berechnung.
      \begin{enumerate}
        \item Prüfe ob $w = $ Kod($M'$) für eine TM $M'$, falls nicht akzeptiert $M$ das Wort.
        \item Falls $w = $ Kod($M'$) für eine TM $M'$, dann wählt $M$ nichtdeterministisch
          ein Wort $x$ über dem Eingabealphabet von $M'$ und simuliert $M'$ deterministisch
          auf $x$. Falls $M'$ auf $x$ hält, akzeptiert $M$ das Wort. Sonst rechnet
          $M$ unendlich lange.
      \end{enumerate}
      Korrektheitsbegründung (schwammig):
      $M$ akzeptiert offensichtlich alle Wörter, welche keine
      TM sind. Falls $w$ eine Kodierung einer TM $M'$ ist, so ist $w \in L^C$ gdw.
      es ein Wort gibt so dass diese TM $M'$ hält. Es gibt eine akzeptierende
      Berechnung von $M$ wo dieses Wort nichtdeterministisch gewählt wird.
      Oder $w \in L$, dann gibt es keine akzeptierende Berechung von $M$.
      Somit $L^C \in \L_{\mathrm{RE}}$
    \end{solutionorbox}}
    \end{parts}

  \question Welche Aussagen sind korrekt?
  ($M$ ist eine MTM, $n\in \mathbb{N}$, $C$ ist eine Konfiguration)
  \begin{checkboxes}
    \CorrectChoice min\{Time$_M(x) \mid x \in \Sigma^n$\}
      + max\{Time$_M(x) \mid x \in \Sigma^n\} \leq 2\cdot$Time$_M(n)$
    \choice Space$_M(n)$ hängt von der Mächtigkeit des
    Arbeitsalphabetes von $M$ ab.
    \choice Space$_M(n)$ hängt von der Mächtigkeit des
    Eingabealphabetes von $M$ ab.
    \CorrectChoice Space$_M(C)$ hängt nicht
    von der Länge des Eingabewortes ab.
  \end{checkboxes}

\end{questions}

\end{document}
