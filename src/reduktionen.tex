\documentclass[a4paper,ngerman,12pt]{exam}
\usepackage{babel}
\usepackage[utf8]{inputenc}
\usepackage[T1]{fontenc}
\usepackage{graphicx}
\usepackage{algpseudocode}
\usepackage{paralist} % kompakte Aufzählungen
\usepackage{textcomp,tikz} %diverses
\usepackage{amsmath,amssymb,amstext,amsthm}
\usepackage{listings}
\usepackage{mathtools}
\usepackage{float}
\usetikzlibrary{calc}
\usetikzlibrary{arrows, automata}
\usepackage{geometry}

\geometry{a4paper, top=3cm, left=2.7cm, right=2.7cm}
\pagestyle{plain}
\renewcommand{\solutiontitle}{\noindent\textbf{Lösung:}\enspace}
\DeclarePairedDelimiter\ceil{\lceil}{\rceil}
\DeclarePairedDelimiter\floor{\lfloor}{\rfloor}
\newcommand\kod{\text{Kod}}
\newcommand\Lu{L_{\mathrm{U}}}
\newcommand\Lh{L_{\mathrm{H}}}
\renewcommand\L{\mathcal{L}}
\newcommand\LR{\mathcal{L}_{\mathrm{R}}}
\newcommand\LDiag{L_{\mathrm{Diag}}}


%\printanswers

\begin{document}
%$ $\\
\noindent Theoretische Informatik \hfill Gruppe 8

\hfill Loris Reiff

\begin{center}
  \bfseries\Large
  Methode der Reduktion \ifprintanswers
  -- Lösungen
  \else
  -- Aufgaben
\fi
\end{center}

\noindent\textit{Alle 4 Aufgaben lassen sich mit der EE-Reduktion lösen und ich habe nur diese Angegeben [Aufgabe 1 und 4] (Ich werde in der nächsten Lektion noch etwas dazu erwähnen).}

\begin{questions}
\question
Zeige dass folgende Sprache
\begin{align*}
  L = \{\kod(M)\#x\#0^i \mid &x \in \{0,1\}^*, i \in \mathbb{N}, M
  \text{ hat mindestens } i+1 \text{ Zustände} \\
  &\text{und während der Berechnung von } M
  \text{ auf } x \text{ wird der } \\
  &i \text{-te Zustand von $M$ min. einmal erreicht}
\end{align*}
keine rekursive Sprache ist.

  \textit{(Aufgabe 5.16 aus dem Buch)}
    \begin{solutionorbox}[18em]
      \begin{proof} $ $\\
      Wir haben in der Vorlesung gesehen, dass $\Lu \not\in \LR$ gilt. Wir zeigen also
      $\Lu \leq_{\mathrm{EE}} L$, was $L \not\in \LR$ impliziert.

      Folgende TM $A$ transformiert eine Eingabe $w$ für $\Lu$ in eine Eingabe für $L$:
      \begin{itemize}
      \item Prüfe, ob $w = \kod(M)\#x$ für eine TM $M$ und ein Wort $x$
      \begin{itemize}
        \item Falls nein, gib $\lambda$ aus.
        \item Falls ja, bestimme $i$, so dass $q_i = q_{\text{accept}}$
        und gib $\kod(M)\#x\#0^i$ aus
      \end{itemize}
      \end{itemize}
      Wir zeigen nun, dass $w \in \Lu \iff A(w) \in L$ gilt:

        Sei $w \in \Lu$:
        \begin{align*}
          w \in \Lu &\implies w=\kod(M)\#x \in \Lu \\
          & \implies w \in L(M) \\
          & \implies A(w) = \kod(M)\#x\#0^i \in L
        \end{align*}
        Wobei im letzten Schritt verwendet wurde, dass $M$ die Berechnung
        in $q_{\mathrm{accept}}$
        beendet, d.h. der $i$-te Zustand wurde min. einmal erreicht. Ausserdem
        hat $M$, genau $i+1$ Zustände, da $q_{\mathrm{accept}}$ der zweit letzte Zustand
        ist (Definition der TM-Kodierung).

        Sei $w \not\in \Lu$: \\
        Fall 1: $w$ hat nicht die Form $\kod(M)\#x$, also gilt $A(w) = \lambda \not\in L$. \\
        Fall 2: $w$ hat die Form $\kod(M)\#x$, also gilt
        \begin{align*}
        x \not\in L(M) \implies A(w) = \kod(M)\#x\#0^i \not\in L
        \end{align*}
        da der $i$-te Zustand, welcher $q_{\mathrm{accept}}$
        entspricht, nie erreicht wird.

        Somit schliessen wir $L \not \in \LR$.
      \end{proof}
  \end{solutionorbox}

\ifprintanswers
\newpage
\fi

\question
   Zeige
      $\Lu^C \leq_{\mathrm{EE}} \LDiag$
    \begin{solutionorbox}[18em]
      \begin{proof} $ $ \\
      Folgende TM $B$ transformiert eine Eingabe $x$ für $\Lu^C$ in eine Eingabe für $\LDiag$:
      \begin{itemize}
      \item Prüfe, ob $x = \kod(M)\#w$ für eine TM $M$ und ein Wort $w$
      \begin{itemize}
        \item Falls nein, konstruiere die TM $M_{\emptyset}$, welche alle Eingaben verwirft.
        \item Falls ja, konstruiere die TM $\widehat{M}$, welche $M$ auf $w$ simuliert und
          die Eingabe ignoriert.
      \end{itemize}
      \item Berechne $i$, so dass $M_i$ die konstruierte TM ist und gib $w_i$ zurück.
      \end{itemize}
      Wir zeigen nun, dass $x \in \Lu^C \iff B(x) \in \LDiag$ gilt:

        Sei $x \in \Lu^C$, dann haben wir zwei Fälle \\
        Fall 1: $x$ hat nicht die Form $\kod(M)\#w$, somit gilt $B(x) \in \LDiag$,
        da $M_i = M_{\emptyset}$
        \hspace*{2.8em} kein Wort akzeptiert, insbesondere nicht $w_i$. \\
        Fall 2: $x$ hat die Form $\kod(M)\#w$:
        \begin{align*}
          x  \in \Lu^C &\implies w\not\in L(M) \\
          &\implies \widehat{M} \text{ verwirft alle Eingaben } \\
          &\implies w_i \not\in \widehat{M} = M_i \\
          &\implies B(x) = w_i \in \LDiag
        \end{align*}

        Sei $x \not\in \Lu^C$, dann gilt:
        \begin{align*}
          x = \kod(M)\#w \not\in \Lu^C &\implies w\in L(M) \\
          &\implies \widehat{M} \text{ akzeptiert alle Eingaben } \\
          &\implies w_i \in \widehat{M} = M_i \\
          &\implies B(x) = w_i \not\in \LDiag
        \end{align*}
      \end{proof}
    \end{solutionorbox}

\ifprintanswers
\newpage
\fi

  \question
   Zeige $\Lh^C \leq_{\mathrm{EE}} \Lu^C$
    \begin{solutionorbox}[22em]
      \begin{proof} $ $ \\
      Folgende TM $C$ transformiert eine Eingabe $x$ für $\Lh^C$ in eine Eingabe für $\Lu^C$:
      \begin{itemize}
        \item Prüfe, ob $x = \kod(M)\#w$ für eine TM $M$ und ein Wort $w$
      \begin{itemize}
        \item Falls nein, gib $\lambda$ zurück
        \item Falls ja, modifiziere die TM $M$ zu $\widehat{M}$, in dem
          alle Transitionen von $q_{\mathrm{reject}}$ nach $q_{\mathrm{accept}}$
          umgeleitet werden und gib $\kod(\widehat{M})\#w$ zurück
      \end{itemize}
      \end{itemize}
      Wir zeigen nun, dass $x \in \Lh^C \iff C(x) \in \Lu^C$ gilt:

        Sei $x \in \Lh^C$: \\
        Fall 1: $x$ hat nicht die Form $\kod(M)\#w$, somit gilt $C(x)=\lambda \in \Lu^C$ \\
        Fall 2:
        \begin{align*}
          x = \kod(M)\#w \in \Lh^C &\implies M \text{ hält nicht auf }w \\
          &\implies \widehat{M} \text{ hält nicht auf } w \\
          &\implies w \not\in L(\widehat{M}) \\
          &\implies C(x) = \kod(\widehat{M})\#w \in \Lu^C
        \end{align*}

        Sei $x \not\in \Lh^C$, dann gilt:
        \begin{align*}
          x = \kod(M)\#w \not\in \Lh^C &\implies M \text{ hält auf } w \\
          &\implies \widehat{M} \text{ akzeptiert } w \\
          &\implies C(x) = \kod(\widehat{M})\#w \not\in \Lu^C
        \end{align*}
      \end{proof}
    \end{solutionorbox}

\ifprintanswers
\newpage
\fi

    \question
    Zeige, dass $L_4 \not\in \LR$ ohne den Satz von Rice zu verwenden.
    \begin{align*}
      L_4 &= \{\kod(M) \mid \text{M akzeptiert } 100\}
    \end{align*}
    \vspace*{-1em}
    \begin{solutionorbox}[22em]
      \begin{proof} $ $ \\
      Wir haben in der Vorlesung gesehen, dass $\Lu \not\in \LR$ gilt. Wir zeigen also
      $\Lu \leq_{\mathrm{EE}} L_4$, was $L_4 \not\in \LR$ impliziert.

      Folgende TM $D$ transformiert eine Eingabe $x$ für $\Lu$ in eine Eingabe für $L_4$:
      \begin{itemize}
        \item Prüfe, ob $x = \kod(M)\#w$ für eine TM $M$ und ein Wort $w$
      \begin{itemize}
        \item Falls nein, gib $\lambda$ zurück
        \item Falls ja, konstruiere die TM $\widehat{M}$, welche $M$ auf $w$ simuliert und
          die Eingabe ignoriert; gib $\kod(\widehat{M})$ zurück.
      \end{itemize}
      \end{itemize}
      Wir zeigen nun, dass $x \in \Lu \iff D(x) \in L_4$ gilt:

        Sei $x \in \Lu$, dann gilt:
        \begin{align*}
          x = \kod(M)\#w \in \Lu &\implies w \in L(M) \\
          &\implies \widehat{M} \text{ akzeptiert alles } \\
          &\implies \widehat{M} \text{ akzeptiert } 100 \\
          &\implies D(x) = \kod(\widehat{M}) \in L_4
        \end{align*}

        Sei $x \not\in \Lh^C$, dann haben wir zwei Fälle: \\
        Fall 1: $x$ hat nicht die Form $\kod(M)\#w$, somit gilt $D(x)=\lambda \not\in L_4$ \\
        Fall 2:
        \begin{align*}
          x = \kod(M)\#w \not\in \Lu &\implies w \not\in L(M) \\
          &\implies \widehat{M} \text{ akzeptiert keine Eingabe} \\
          &\implies 100 \not \in L(\widehat{M}) \\
          &\implies D(x) = \kod(\widehat{M}) \not\in L_4
        \end{align*}
      \end{proof}
    \end{solutionorbox}
\end{questions}

\end{document}
