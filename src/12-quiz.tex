\documentclass[a4paper,ngerman,12pt]{exam}
\usepackage{babel}
\usepackage[utf8]{inputenc}
\usepackage[T1]{fontenc}
\usepackage{graphicx}
\usepackage{algpseudocode}
\usepackage{csquotes} % Anführungszeichen
\usepackage{paralist} % kompakte Aufzählungen
\usepackage{textcomp,tikz} %diverses
\usepackage{amsmath,amssymb,amstext,amsthm}
\usepackage{listings}
\usepackage{mathtools}
\usepackage{mdframed} % Boxen
\usepackage{geometry}
\usepackage{float}
\usepackage{tikz}
\usetikzlibrary{calc}
\usetikzlibrary{arrows, automata}

\geometry{a4paper, top=3cm, left=2.7cm, right=2.7cm}
\pagestyle{plain}
\renewcommand{\solutiontitle}{\noindent\textbf{Lösung:}\enspace}
\DeclarePairedDelimiter\ceil{\lceil}{\rceil}
\DeclarePairedDelimiter\floor{\lfloor}{\rfloor}
\renewcommand\L{\mathcal{L}}
\newcommand\ST{\Sigma_{\mathrm{T}}}
\newcommand\SN{\Sigma_{\mathrm{N}}}

%\printanswers

\begin{document}
\noindent Theoretische Informatik \hfill Gruppe 8
\begin{center}
  \bfseries\Large
  Quiz 12 \ifprintanswers
  -- Lösungen
\fi
\end{center}


\begin{questions}
\question
  Zeige, dass $L = \{ww \mid w \in \{0,1\}^*\}$ keine kontextfreie Sprache ist.
  \begin{solutionorbox}[18em] $ $ \\
    \textit{Proof-Sketch} \\
    Nehmen wir an $L$ sei kontextfrei, d.h. Pumping-Lemma gilt für $L$.
    Betrachten wir das Wort
    \begin{align*}
      z = 0^{n_L}1^{n_L}0^{n_L}1^{n_L}
    \end{align*}
    Offensichtlich gilt $|z| = 4n_L > n_L$. Es gibt also eine Zerlegung
    $z = uvwxy$ mit den 3 Eigenschaften. Mit (ii) $uwx \leq n_L$ können
    wir folgern, dass $u$ und $v$ nur zwei unterschiedliche Buchstaben
    aus dem ersten und dem \textit{repertierten Wort} haben kann.
    Mit (i) $|vx| \geq 1$ erhalten wir einen Widerspruch in (iii), denn
    $uv^2wx^2y$ ist nicht in $L$.
  \end{solutionorbox}

  \question
  Entwerfe den nichtdeterministischen Kellerautomaten für
  \begin{align*}
    L = \{ww^{\mathrm{R}} \mid w \in \{a,b\}^*\}
  \end{align*}
  \begin{solutionorbox}[20em] $ $ \\
    $M = (\{q_0,q_1\}, \{a,b\},\{a,b,Z_0\}, \delta, q_0, Z_0)$ mit
    der
    \begin{align*}
      \delta(q_0, a, Z_0) &= \{(q_0, Z_0a),(q_1, Z_0a)\} \\
      \delta(q_0, b, Z_0) &= \{(q_0, Z_0b),(q_1, Z_0b)\} \\
      \delta(q_0, a, a) &= \{(q_0, aa),(q_1, aa)\} \\
      \delta(q_0, a, b) &= \{(q_0, ab),(q_1, ab)\} \\
      \delta(q_0, b, a) &= \{(q_0, ba),(q_1, ba)\} \\
      \delta(q_0, b, b) &= \{(q_0, bb),(q_1, bb)\} \\
      \delta(q_0, \lambda, Z_0) &= \{(q_1, \lambda)\} \\
      \delta(q_1, a, a) &= \{q_1, \lambda\} \\
      \delta(q_1, b, b) &= \{q_1, \lambda\} \\
      \delta(q_1, \lambda, Z_0) &= \{(q_1, \lambda)\}
    \end{align*}
  \end{solutionorbox}
\end{questions}

\end{document}
