\documentclass[a4paper,ngerman,12pt]{exam}
\usepackage{babel}
\usepackage[utf8]{inputenc}
\usepackage[T1]{fontenc}
\usepackage{graphicx}
\usepackage{algpseudocode}
\usepackage{csquotes} % Anführungszeichen
\usepackage{paralist} % kompakte Aufzählungen
\usepackage{textcomp,tikz} %diverses
\usepackage{amsmath,amssymb,amstext,amsthm}
\usepackage{listings}
\usepackage{mathtools}
\usepackage{mdframed} % Boxen
\usepackage{geometry}
\usepackage{float}
\usepackage{tikz}
\usetikzlibrary{calc}
\usetikzlibrary{arrows, automata}

\geometry{a4paper, top=3cm, left=2.7cm, right=2.7cm}
\pagestyle{plain}
\renewcommand{\solutiontitle}{\noindent\textbf{Lösung:}\enspace}
\DeclarePairedDelimiter\ceil{\lceil}{\rceil}
\DeclarePairedDelimiter\floor{\lfloor}{\rfloor}
\renewcommand\L{\mathcal{L}}
\newcommand\ST{\Sigma_{\mathrm{T}}}
\newcommand\SN{\Sigma_{\mathrm{N}}}

%\printanswers

\begin{document}
\noindent Theoretische Informatik \hfill Gruppe 8 \\
\mbox{}\hfill Loris Reiff
\begin{center}
  \bfseries\Large
  Quiz 9 \ifprintanswers
  -- Lösungen
\fi
\end{center}

\begin{questions}
\question Vereinfache folgende reguläre Ausdrücke:
  \begin{parts}
    \part $\lambda + a(\lambda + a + ba)^* b$
  \uplevel{\begin{solutionorbox}[2.5em]
    $\lambda + a(a + ba)^* b$
  \end{solutionorbox}}
    \part $aba + aba(\lambda + ba + baa)^*(\lambda + ba + baa)$
  \uplevel{\begin{solutionorbox}[2.5em]
    $aba(ba + baa)^*$
  \end{solutionorbox}}
    \part Welche Wörter gehören zu welcher Sprache?
    \begin{align*}
      L_a &= L(\lambda + a(\lambda + a + ba)^* b) \\
      L_b &= L(aba + aba(\lambda + ba + baa)^*(\lambda + ba + baa))
    \end{align*}
    \begin{itemize}
      \item $\lambda$
      \item $ab$
      \item $abab$
      \item $ababa$
      \item $abababaa$
    \end{itemize}
   \begin{solution} \\
    $\lambda \in L_a$, \hspace{2.9em}
      $ab \in L_a$, \\
      $abab \in L_a$,\hspace{2em}
      $ababa \in L_b$ \\
      $abababaa \in L_b$
  \end{solution}
  \end{parts}

\question Entwerfe eine Grammatik für die folgende Sprache
  \begin{align*}
    L = \{w \in \{0,1\}^* \mid |w|_1 \geq 2\}
  \end{align*}
  und gib eine Ableitung des Wortes $00101$ an.
\begin{solutionorbox}[8em]
  $G = (\SN, \ST, P, S)$ mit
  \begin{enumerate}[(i)]
    \item $\SN = \{S, X, Y\}$,
    \item $\ST = \{0,1\}$,
    \item $P = \{S \to 0S \mid 1X,$ \\
      \hspace*{2.4em}$X \to 0X \mid 1Y,$ \\
      \hspace*{2.4em}$Y \to 0Y \mid 1Y \mid \lambda\}$
  \end{enumerate}

  $S \implies 0S \implies 00S \implies 001X \implies 0010X \implies 00101Y
  \implies 00101$
\end{solutionorbox}

\question Sei
  $G = (\{S,X\}, \{0,1\}, P, S)$ mit
  \begin{align*}
    P = \{S \to 1X1, X \to \lambda, X \to XS, X \to X0, X \to 1, X \to XX, X0 \to 0X\}
  \end{align*}
  \begin{parts}
    \part
Bestimme die generierte Sprache.
    \uplevel{\begin{solutionorbox}[5em]
      $\{1w1 \mid w \in \{0,1\}^*\}$
    \end{solutionorbox}}
    \part Bestimme eine reguläre (Typ-3) Grammatik, welche äquivalent ist.
    Also die gleiche Sprache generiert.
    \uplevel{\begin{solutionorbox}[18em]
  $G = (\SN, \ST, P, S)$ mit
  \begin{enumerate}[(i)]
    \item $\SN = \{S, X\}$,
    \item $\ST = \{0,1\}$,
    \item
        $P = \{S \to 1X,$ \\
        \hspace*{2.1em} $X \to 0X \mid 1X \mid 1 \}$
  \end{enumerate}
    \end{solutionorbox}}
  \end{parts}


\end{questions}

\end{document}
