\documentclass[a4paper,ngerman,12pt]{exam}
\usepackage{babel}
\usepackage[utf8]{inputenc}
\usepackage[T1]{fontenc}
\usepackage{graphicx}
\usepackage{algpseudocode}
\usepackage{geometry}
\usepackage{csquotes} % Anführungszeichen
\usepackage{paralist} % kompakte Aufzählungen
\usepackage{textcomp,tikz} %diverses
\usepackage{amsmath,amssymb,amstext,amsthm}
\usepackage{listings}
\usepackage{mathtools}
\usepackage{mdframed} % Boxen
\usepackage{float}
\usepackage{tikz}
\usetikzlibrary{calc}
\usetikzlibrary{arrows, automata}

\geometry{a4paper, top=3cm, left=2.7cm, right=2.7cm}
\pagestyle{plain}
\renewcommand{\solutiontitle}{\noindent\textbf{Lösung:}\enspace}
\DeclarePairedDelimiter\ceil{\lceil}{\rceil}
\DeclarePairedDelimiter\floor{\lfloor}{\rfloor}

%\printanswers

\begin{document}
\noindent Theoretische Informatik \hfill Gruppe 8 \\
\mbox{}\hfill Loris Reiff
\begin{center}
  \bfseries\Large
  Quiz 5\ifprintanswers
  -- Lösungen
\fi
\end{center}

\begin{questions}
  \question
  Welche Aussage ist korrekt?
  \begin{checkboxes}
    \CorrectChoice $\mathcal{L}_{\mathrm{EA}} \subsetneq
          \mathcal{L}_{\mathrm{R}}$
    \choice $\mathcal{L}_{\mathrm{EA}} =
          \mathcal{L}_{\mathrm{R}}$
    \choice $\mathcal{L}_{\mathrm{EA}}\supsetneq
          \mathcal{L}_{\mathrm{R}}$
  \end{checkboxes}

  \question
  Welche Aussagen sind korrekt ($M_1, M_2$ sind TM)?
  \begin{checkboxes}
    \choice $L(M_1) = L(M_2) \implies M_1 = M_2$
    \choice $L(M_1) = L(M_2) \iff M_1 = M_2$
    \CorrectChoice $L(M_1) = L(M_2) \impliedby M_1 = M_2$
  \end{checkboxes}

\question \textit{(Wiederholung Kapitel 2)} \\
  Sei $w = 1^{2^{3\cdot n^2}} \in \{0,1\}^*$ für alle $n \in \mathbb{N}$.
Geben Sie eine möglichst gute obere
Schranke für die Kolmogorov-Komplexität von $w_n$ an, gemessen in der Länge von
$w_n$.

\begin{solutionorbox}[29em]
  Wir geben zunächst für jedes $n\in \mathbb{N}$ ein Programm an, welches
  $w_n$ erzeugt
\begin{lstlisting}[language=Pascal, mathescape=true]
begin
  x := $n$;
  x := 2^(3 * x*x);
  for i:=1 to x do
    write(1);
end;
\end{lstlisting}
Der einzige Teil des Maschinencodes dieses Programms, der von $w_n$ abhängt, ist
die Darstellung von $n$ in der zweiten Zeile.
Der restliche Programmcode hat eine konstante Länge.
Also ist die binäre Länge dieses Programms $\ceil{\log_2 (n + 1)} + c$ für
eine Konstante c.

Damit lässt sich die Kolmogorov-Komplexität von $w_n$ von oben abschätzen durch
  \begin{align*}
    K(w_n) \leq \ceil{\log_2 (n + 1)} + c
  \end{align*}
  Die Länge von $w_n$ ist: $|w_n| = 2^{3 \cdot n^2} \iff \sqrt{\log_2 |w_n|/3} = n$,
  somit
  \begin{align*}
    K(w_n) &\leq \frac{1}{2}\log_2 (\log_2 |w_n|/3 ) + c' \\
      &\leq \frac{1}{2}\log_2 \log_2 |w_n| + c''
  \end{align*}
  für Konstanten $c', c''$.
    \end{solutionorbox}

\end{questions}

\end{document}
