\documentclass[a4paper,ngerman,12pt]{exam}
\usepackage{babel}
\usepackage[utf8]{inputenc}
\usepackage[T1]{fontenc}
\usepackage{graphicx}
\usepackage{algpseudocode}
\usepackage{geometry}
\usepackage{csquotes} % Anführungszeichen
\usepackage{paralist} % kompakte Aufzählungen
\usepackage{amsmath,textcomp,tikz} %diverses
\usepackage{mdframed} % Boxen

\geometry{a4paper, top=3cm, left=2.7cm, right=2.7cm}
\pagestyle{plain}
\renewcommand{\solutiontitle}{\noindent\textbf{Lösung:}\enspace}

%\printanswers

\begin{document}
\noindent Theoretische Informatik \hfill Gruppe 8 \\
\mbox{}\hfill Loris Reiff
\begin{center}
  \bfseries\Large
  Quiz 1 \ifprintanswers
  -- Lösungen
\fi
\end{center}

\begin{questions}
  \question Gibt es eine nichtleere endliche Sprache $L \neq \{\lambda\}$ über dem Alphabet
  $\{a,b\}$, die die Bedingung $L^2 = L$ erfüllt?

\begin{oneparcheckboxes}
\choice Ja
\CorrectChoice Nein
\end{oneparcheckboxes}
\begin{solution}
Es gibt keine solche Sprache!
Angenommen, es gäbe eine nichtleere endliche Sprache $L \neq \{\lambda\}$, so dass $L^2 = L$,
dann gäbe es ein Wort $w \in L$ mit $|w| = \max\{|v| \mid v \in L\}$. Da
$L \neq \{\lambda\}$ gilt, folgt $|w| \geq 1$. Aus unserer Annahme $L^2 = L$ folgt des
Weiteren, dass $w^2 \in L$ gelten muss.
Dies ist aber ein Widerspruch zur Annahme, dass $w$ ein Wort
maximaler Länge ist ($|w^2| > |w|$)!
\end{solution}

\question
  Sei $L_1 = \{\{0\}^*\{1\}^*\}^*$ und
      $L_2 = \{\{0, 1\}^3\}^*$. Welche Aussage ist korrekt?

\begin{oneparcheckboxes}
\choice $L_1 = L_2$
\CorrectChoice $L_1 \neq L_2$
\end{oneparcheckboxes}
  \begin{solution}
    Die Länge aller Wörter in $L_2$ sind ein Vielfaches von 3, folglich sind
    $L_1$ und $L_2$ nicht gleich, da z.B $0 \in L_1$ aber $0 \not\in L_2$.
  \end{solution}

\question
Seien $L_1$, $L_2$ und $L_3$ Sprachen über einem Alphabet $\Sigma$. Dann gilt
  \begin{align*}
    L_1L_2 \cup L_1L_3 = L_1(L_2 \cup L_3)
  \end{align*}
\begin{oneparcheckboxes}
\CorrectChoice Wahr
\choice Falsch
\end{oneparcheckboxes}
  \begin{solution}
    Siehe Lemma 2.1 auf Seite 21 im Buch.
  \end{solution}

\question
Seien $L_1$, $L_2$ und $L_3$ Sprachen über einem Alphabet $\Sigma$. Dann gilt
  \begin{align*}
    L_1L_2 \cap L_1L_3 = L_1(L_2 \cap L_3)
  \end{align*}
\begin{oneparcheckboxes}
\choice Wahr
\CorrectChoice Falsch
\end{oneparcheckboxes}
  \begin{solution}
    Gegenbeispiel (über $\Sigma_{\mathrm{Bool}}$): $L_1 = \{\lambda, 1\}$,
    $L_2 = \{0\}$ und $L_3 = \{10\}$. Somit
    $L_1(L_2 \cap L_3) = \emptyset$ und $L_1L_2 \cap L_1L_3 = \{10\}$
  \end{solution}


\question
Wir betrachten die Sprache
  \begin{align*}
    L = \{p, pq, pp, pqp, pqqp\}
  \end{align*}
  Gibt es zwei Sprachen $L_1 \neq \{\lambda\}$ und $L_2 \neq \{\lambda\}$ über dem
  Alphabet $\Sigma = \{p,q\}$, so dass $L = L_1 \cdot L_2$?
  Falls ja, bestimme $L_1$ und $L_2$. Falls nein, begründe warum solche Sprachen nicht
  existieren können.
  \begin{solutionorbox}[6em]
    $L_1 = \{p,pq\}$ und $L_2 = \{\lambda, p, qp\}$ somit
    \begin{align*}
      L_1 \cdot L_2 &= \{p, pp, pqp, pq, pqp, pqqp\} \\
        &= \{p, pq, pp, pqp, pqqp\} = L
    \end{align*}
  \end{solutionorbox}

\question
Schreibe einen Algorithmus $\mathcal{A}$ (in Pseudocode), welcher folgendes
Entscheidungsproblem löst:
$(\Sigma_{10}, \{x \in (\Sigma_{10})^* \mid x \text{ ist durch 3 teilbar}\})$

Alternative Darstellung:

\textit{Eigabe:} $x \in \left(\Sigma_{10}\right)^*$ \\
\textit{Ausgabe:} Ja, falls $x$ durch 3 teilbar ist. Nein, sonst.
  \begin{solutionorbox}[8em]
\algtext*{EndFunction}
  \begin{algorithmic}[1]
    \Function{$\mathcal{A}$}{$x$}
        \State \Return{x \textbf{mod} 3 = 0}
    \EndFunction
  \end{algorithmic}
  \end{solutionorbox}

\end{questions}
\end{document}
