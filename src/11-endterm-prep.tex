\documentclass[a4paper,ngerman,12pt]{exam}
\usepackage{babel}
\usepackage[utf8]{inputenc}
\usepackage[T1]{fontenc}
\usepackage{graphicx}
\usepackage{geometry}
\usepackage{csquotes} % Anführungszeichen
\usepackage{paralist} % kompakte Aufzählungen
\usepackage{textcomp,tikz} %diverses
\usepackage{amsmath,amssymb,amstext,amsthm}
\usepackage{mathtools}
\usepackage{tikz}
\usetikzlibrary{calc}
\usetikzlibrary{shapes, arrows, positioning,automata}

\geometry{a4paper, top=3cm, left=2.7cm, right=2.7cm, bottom=2.5cm}
\pagestyle{plain}
\renewcommand{\solutiontitle}{\noindent\textbf{Lösung:}\enspace}
\DeclarePairedDelimiter\ceil{\lceil}{\rceil}
\DeclarePairedDelimiter\floor{\lfloor}{\rfloor}
\newcommand\kod{\text{Kod}}
\renewcommand\L{\mathcal{L}}
\newcommand\LR{\mathcal{L}_{\mathrm{R}}}
\newcommand\Lu{L_{\mathrm{U}}}
\newcommand\Lns{L_{\not\subseteq}}
\newcommand\Lul{L_{\mathrm{U}, \lambda}}
\newcommand\EE{\leq_{\mathrm{EE}}}
\newcommand\Lint{L_{\mathrm{intersect}}}
\newcommand\nmSAT{non-3-monotone-3SAT}

%\printanswers

\begin{document}
\noindent Theoretische Informatik \hfill Gruppe 8 \\
\mbox{}\hfill Loris Reiff
\begin{center}
  \bfseries\Large
  Endterm Vorbereitung \ifprintanswers
  -- Lösungen
\fi
\end{center}


\begin{questions}
\question
Betrachte
  \begin{align*}
    \Lns &= \{\kod(M_1)\#\kod(M_2) \mid M_1 \text{ und } M_2
    \text{ sind TMs mit } L(M_1) \not\subseteq L(M_2) \} \\
    \Lul &= \{\kod(M) \mid M \text{ ist eine TM und } \lambda \in L(M)\} \\
    \Lint &= \{\kod(M_1)\#\kod(M_2)\#w \mid M_1 \text{ und } M_2
    \text{ sind TMs mit } w \in L(M_1) \cap L(M_2)\}
  \end{align*}
  \begin{parts}
    \part Zeige $\Lint \not\in \LR$
\ifprintanswers
    \uplevel{\begin{solution}
      Wir haben in der Vorlesung gesehen, dass $\Lu \not\in \LR$. Wir zeigen
      nun $\Lu \EE \Lint$, was $\Lint \not\in \LR$ impliziert.

      Folgende TM $A$ transformiert eine Eingabe $x$ für $\Lu$ in eine Eingabe
      für $\Lint$:
      \begin{itemize}
        \item Prüfe, ob die Eingabe die Form $\kod(M)\#w$ für eine TM $M$ hat.
        \begin{itemize}
          \item Falls ja:
            Gib $\kod(M)\#\kod(M)\#w$ zurück.
          \item Falls nein: gib $\lambda$ aus
        \end{itemize}
      \end{itemize}

      Wir haben offensichtlich
      \begin{align*}
        \kod(M)\#w \in \Lu &\iff w \in L(M) \\
        &\iff w \in L(M) \cap L(M) \\
        & \iff \kod(M)\#\kod(M)\#w \in \Lint
      \end{align*}
      und für $x \neq \kod(M)$ haben wir $x \not\in \Lu \implies \lambda \not\in \Lint$.

      es gilt also $x \in \Lu \iff A(x) \in \Lint$.


    \end{solution}}
\fi
\ifprintanswers
\newpage
\fi
    \part Zeige $\Lint \EE \Lul$
\ifprintanswers
    \uplevel{\begin{solution}
      Folgende TM $B$ transformiert eine Eingabe $x$ für $\Lint$ in eine Eingabe
      für $\Lul$:
      \begin{itemize}
        \item Prüfe, ob die Eingabe die Form $\kod(M_1)\#\kod(M_2)\#w$ für TM $M_1$ und $M_2$
          hat.
        \begin{itemize}
          \item Falls ja: Konstruiere die TM $\overline{M}$, welche die Eingabe ignoriert
            und $w$ parallel auf $M_1$ und $M_2$ simuliert und nur akzeptiert, falls
            beide TM akzeptieren, d.h. wenn $M_1$ oder $M_2$ verwirft, verwirft auch
            $\overline{M}$.
          \item Falls nein: gib $\lambda$ aus.
        \end{itemize}
      \end{itemize}

      Wir zeigen nun: $x \in \Lint \iff B(x) \in \Lul$:

      ''$\Rightarrow$'': \\
      Sei $x \in \Lint$, d.h. $x = \kod(M_1)\#\kod(M_2)\#w$
      für TM $M_1$ und $M_2$ mit $w \in L(M_1) \cap L(M_2) \implies
      w \in L(M_1) \wedge w \in L(M_2)$. Folglich hält $\overline{M}$ gemäss Konstruktion
      für jede Eingabe in $q_{\text{accept}}$, wir haben insbesondere
      $\lambda \in L(\overline{M})$, also $B(x) \in \Lul$.

      ''$\Leftarrow$'': \\
      Sei $x \not\in \Lint$, wir betrachten zwei Fälle:
        \begin{itemize}
         \item
            $x$ hat nicht die Form $\kod(M_1)\#\kod(M_2)\#w$ für TM $M_1$ und $M_2$,
            dann ist
              $B(x) = \lambda \not\in \Lul$.
            \item $x = \kod(M_1)\#\kod(M_2)\#w$ für TM $M_1$ und $M_2$,
              d.h. $w \not\in L(M_1) \cap L(M_2)$
              das impliziert $w \not\in L(M_1) \vee w \not\in L(M_2)$,
              somit akzeptiert $\overline{M}$ keine Eingabe, also
              $\lambda \not\in L(\overline{M}) = \emptyset$ und somit haben wir
              $B(x) \not\in \Lul$
        \end{itemize}

    \end{solution}}
\fi
\ifprintanswers
\newpage
\fi
    \part Zeige $\Lul \EE \Lns$
\ifprintanswers
    \uplevel{\begin{solution}
      Folgende TM $C$ transformiert eine Eingabe $x$ für $\Lul$ in eine Eingabe
      für $\Lns$:
      \begin{itemize}
        \item Prüfe, ob die Eingabe die Form $\kod(M)$ für eine TM $M$ hat.
        \begin{itemize}
          \item Falls ja: Konstruiere die TM $\overline{M}$, welche die Eingabe ignoriert und
            $\lambda$ auf $M$ simuliert.
            Konstruiere des Weiter die TM $M_\emptyset$, welche alle Eingaben verwirft.
            Gib $\kod(\overline{M})\#\kod(M_{\emptyset})$ zurück.
          \item Falls nein: gib $\kod(M_\emptyset)\#\kod(M_\emptyset)$ aus
        \end{itemize}
      \end{itemize}

      Wir zeigen nun: $x \in \Lul \iff C(x) \in \Lns$:

      Sei $x \in \Lul$, d.h. $x = \kod(M)$ für eine TM $M$ mit $\lambda \in L(M)$.
      Es gilt also $L(\overline{M}) = \Sigma^*$, und da
      $\Sigma^* \not\subseteq \emptyset = L(M_\emptyset)$ ist $C(x) \in \Lns$

      Sei $x \not\in \Lul$, wir betrachten zwei Fälle:
        \begin{itemize}
         \item
            $x$ hat nicht die Form $\kod(M)$ für eine TM $M$, dann ist \\
              $C(x) = \kod(M_\emptyset)\#\kod(M_\emptyset)$ und da
              $\emptyset \subseteq \emptyset$ ist $C(x) \not\in \Lns$
            \item $x = \kod(M)$ für eine TM $M$, dann gilt $\lambda \not\in L(M)$,
              somit akzeptiert $\overline{M}$ keine Eingabe, also
              $L(\overline{M}) = \emptyset$ und somit haben wir \\
              $C(x) = \kod(\overline{M})\#\kod(M_{\emptyset}) \not\in \Lns$,
              da $\emptyset \subseteq \emptyset$.
        \end{itemize}

    \end{solution}}
\fi
  \end{parts}

\ifprintanswers
\newpage
\fi

\question
  \begin{parts}
    \part Entwerfe eine reguläre Grammatik für folgende Sprache:
    \begin{align*}
      L_a = \{w \in \{a,b\} \mid 1 \leq |w|_a \leq 2 \text{ oder } w
      \text{ enthält das Teilwort } baaab\}
    \end{align*}
    Begründe deinen Entwurf kurz.
\ifprintanswers
    \uplevel{\begin{solution}
      Die Sprache ist äquivalent zu $L_1 \cup L_2$ mit
      \begin{align*}
        L_1 &= \{w \in \{a,b\} \mid 1 \leq |w|_a \leq 2\} \\
        L_2 &= \{w \in \{a,b\} \mid w \text{ enthält das Teilwort } baaab\}
      \end{align*}
      Wir konstruieren zuerst die Grammatiken für $L_1$ und $L_2$. \\
      Sei $G_1 = (\{S_1, A_1, A_2\}, \{a,b\}, P_1, S_1)$ mit
      \begin{align*}
        P_1 = \{&S_1 \to bS_1 \mid aA_1, \\
                &A_1 \to bA_1 \mid aA_2 \mid \lambda, \\
                &A_2 \to bA_2 \mid \lambda
                \}
      \end{align*}
      Die Idee ist hierbei die Anzahl
      der bereits generierten $a$'s über die Nichtterminale zu merken.
      Es gilt offensichtlich $L(G_2) = L_1$.

      Und sei $G_2 = (\{S_2, E\}, \{a,b\}, P_2, S_2)$ mit
      \begin{align*}
        P_2 = \{&S_2 \to aS_2 \mid bS_2 \mid baaabE, \\
                &E \to aE \mid bE \mid \lambda
                \}
      \end{align*}
      Hier ist die Idee mit den Regeln $S_2 \to aS_2 \mid bS_2$ einen beliebigen
      Präfix über $\{a,b\}$ zu generieren. Mit $S_2 \to baaabE$ stellen wir sicher,
      dass das generierte Wort das Teilwort $baaab$ enthält. Und mit
      $E \to aE \mid bE \mid \lambda$ kann ein beliebiger Suffix generiert werden.

      Wir konstruieren nun die Vereinigung der zwei Grammatiken: \\
      $G_a = (\{S, S_1, S_2, A_1, A_2, E\}, \{a,b\}, P, S)$ mit
      \begin{align*}
        P_1 = \{&S \to S_1 \mid S_2, \\
                &S_1 \to bS_1 \mid aA_1, \\
                &A_1 \to bA_1 \mid aA_2 \mid \lambda, \\
                &A_2 \to bA_2 \mid \lambda, \\
                &S_2 \to aS_2 \mid bS_2 \mid baaabE, \\
                &E \to aE \mid bE \mid \lambda
                \}
      \end{align*}
    \end{solution}}
\fi

\ifprintanswers
\newpage
\fi

    \part Entwerfe eine Grammatik für folgende Sprache:
    \begin{align*}
      L_b = \{a^i b^j c^k \mid i,j,k \in \mathbb{N}, i=j+k\}
    \end{align*}
    Begründe deinen Entwurf kurz.
\ifprintanswers
  \uplevel{\begin{solution}
Sei $G_b = (\{S, X, Y, Z\}, \{a,b,c\}, P_b, S)$ mit
    \begin{align*}
      P_b = \{&S \to aSX \mid Y \mid Z, \\
              &YX \to bY \mid bZ, \\
              &ZX \to cZ, \\
              &Z \to \lambda\}
    \end{align*}
    Die Idee ist wie folgt: $S \to aSX$ generiert eine beliebige Anzahl $a$'s und
    für jedes $a$ ein $X$, wobei die $a$'s von den $X$ durch ein $S$ getrennt bleiben.
    $S$ kann danach mit $S \to Y$ bzw. $S \to Z$ zu einem Cursor für $b$'s bzw. $c$'s
    umgewandelt werden. Die Regel $YX \to bY$ erlaubt das Umwandeln von $X$ zu $b$'s.
    Mit der Regel $YX \to bZ$ wird zustätzlich der Cursor in den Cursor für $c$ umgewandelt.
    Mit $ZX \to cZ$ können dann die restlichen $X$ zu $c$'s umgewandelt werden.
\end{solution}}
\fi


    \part Betrachte die Grammatik $G_c = (\{S, X, Y\}, \{0,1\}, P_c, S)$ mit
    \begin{align*}
      P_c = \{&S \to XY, \\
              &X \to 0X1 \mid \lambda, \\
              &Y \to 1Y1 \mid X\}
    \end{align*}
    Gib die erzeugte Sprache der Grammatik $G_c$ an und begründe kurz.

\ifprintanswers
    \uplevel{\begin{solution} $ $\\
    Wir bemerken, die Regeln für $X$ erzeugen $\{0^i1^i \mid i \in \mathbb{N}\}$.
    Des Weitern sehen wir, dass die Regeln für $Y$ folgende Sprache
    $\{1^i0^{j}1^{i+j} \mid i,j \in \mathbb{N}\}$
    generieren. Mit $S \to XY$ haben wir also
    $\{0^i1^i \mid i \in \mathbb{N}\} \cdot \{1^i0^{j}1^{i+j} \mid i,j \in \mathbb{N}\}$ und
    somit:
    \begin{align*}
        \{0^i1^i \mid i \in \mathbb{N}\} \cdot \{1^j0^{k}1^{j+k} \mid j,k \in \mathbb{N}\}
        &= \{0^i1^i1^j0^k1^{j+k} \mid i,j,k \in \mathbb{N}\} \\
        &= \{0^i1^{i+j}0^k1^{j+k} \mid i,j,k \in \mathbb{N}\} \\
        &= \{0^i1^{m}0^k1^{l} \mid i+l = m+k, i \leq m\}
    \end{align*}
      Die erzeugte Sprache ist also:
      $L_c = \{0^i1^{j}0^k1^{l} \mid i+l = j+k, i \leq j\}$
    \end{solution}}
\fi
  \end{parts}

\ifprintanswers
\newpage
\fi

  \question
  %Sei $f: \mathbb{N} \to \mathbb{N}$ eine monoton steigende Funktion mit $f(n) \geq n$
  %für alle $n \in \mathbb{N}$
%
  Seien $L \in$ NTIME$(f)$ und $L' \in$ TIME($f$). Zeige, dass dann
  $L - L' \in$ NTIME($f$) gilt.
\ifprintanswers
  \uplevel{\begin{solution}
    Seien $L \in \text{NTIME}(f)$ und $L' \in $TIME$(f)$.
    Dann existieren eine nichtdeterministische
    $k_1$-Band-Turingmaschine $M_1$ für $L$ und eine deterministische
    $k_2$-Band-Turingmaschine
    $M_2$ für $L'$ mit $\text{Time}_{M_1}(n)$, $\text{Time}_{M_2}(n) \in 
    \mathcal{O}(f(n))$. Wir konstruieren hieraus eine
    nichtdeterministische ($k_1 + k_2$)-Band-TM $M$ für $L - L'$ mit
    $\mathcal{O}(f(n))$ Zeitbedarf wie folgt.
    Zunächst simuliert $M$ die Arbeit von $M_2$ auf der Eingabe $w$ der Länge $n$ auf den
    Arbeitsbändern $k_1 + 1$ bis $k_1 + k_2$. Falls $M_2$ den akzeptierenden Zustand erreicht
    hat, dann gilt $w \in L'$, also $w \not\in L - L'$,
    also verwirft $M$ die Eingabe. Falls $M_2$ den
    verwerfenden Zustand erreicht, dann gilt $w \not\in L'$.
    In diesem Fall setzt $M$ den Lesekopf
    auf dem Eingabeband zurück an den Anfang und startet eine Simulation von $M_1$ auf
    $w$ auf den ersten $k_1$ Arbeitsbändern. Falls $M_1$ das Wort
    $w$ akzeptiert, dann akzeptiert auch $M$.
    Die Zeitkomplexität von $M$ lässt sich wie folgt abschätzen. Die Simulation von $M_2$
    benötigt offenbar $\mathcal{O}(f(n))$ Schritte,
    das Zurücksetzen des Lesekopfes dann noch einmal
    höchstens $\mathcal{O}(f(n))$ Schritte.
    Wenn das Wort $w$ von $M_1$ akzeptiert wird, gibt es nach
    Definition der nichtdeterministischen Zeitkomplexität auch eine Berechnung, in der die
    Simulation von $M_1$ in $\mathcal{O}(f(n))$ Zeit
    durchgeführt wird. Also gilt Time$_M(n) \in \mathcal{O}(f(n))$.
  \end{solution}}
\fi

\ifprintanswers
\newpage
\fi

  \question
  \begin{parts}
    \part{} Sei VIERFACH-SAT die Menge aller KNF-Formeln, welche vier
    erfüllende Belegungen hat. \\
    Zeige, dass VIERFACH-SAT NP-vollständig ist.

\ifprintanswers
  \uplevel{\begin{solution}
    Es gilt VIERFACH-SAT $\in$ NP, denn eine NTM kann die vier Belegungen der Formel
    nichtdeterministisch erraten und prüfen, ob sie erfüllt werden.
    Dies ist offensichtlich in polynomieller Zeit möglich.
    (Es kann natürlich auch mit einem polynomiellen Verifizierer argumentiert werden,
    da VC = NP.)

    Wir zeigen nun SAT $\leq_p$ VIERFACH-SAT, was die Behauptung impliziert:

    Sei $F = F_1 \wedge F_2 \wedge \dots \wedge F_m$ eine Formel in KNF über die
    Variablen $X = \{x_1, \dots, x_n\}$.
    Wir konstruieren aus $F$ eine Eingabe $C$ für das VIERFACH-SAT Problem, so dass
    \begin{align*}
        F \in \text{SAT} \iff C \in \text{VIERFACH-SAT}
    \end{align*}
    Dies tun wir wie folgt: Seien $y_1,y_2,y_3 \not\in X$ drei neue Variablen,
    dann definieren wir
    $C = F_1 \wedge F_2 \wedge \dots \wedge F_m \wedge (y_1 \vee y_2 \vee y_3)$. \\
    Diese Konstruktion von $C$ ist offensichtlich in polynomieller Zeit möglich.

    Wir zeigen nun $F \in \text{SAT} \iff C \in \text{VIERFACH-SAT}$:

    ''$\Rightarrow$'': \\
    Sei $F \in$ SAT, es gibt also eine Belegung $\alpha$, welche die Klauseln
    $F_1, F_2, \dots, F_m$ erfüllt. Wir können $\alpha$ zu vier Belegungen
    $\widehat{\alpha}_1,\widehat{\alpha}_2,\widehat{\alpha}_3,\widehat{\alpha}_4$
    auf $X \cup \{y_1, y_2, y_3\}$ erweitern. Dies tun wir wie folgt:
    Wir setzen $\widehat{\alpha}_i(x) = \alpha(x)$ für alle $x \in X$ und
    $i \in \{1,2,3,4\}$ und
    \begin{itemize}
      \item $\widehat{\alpha}_1(y_1) = \widehat{\alpha}_1(y_2) = \widehat{\alpha}_1(y_3) = 1$
      \item $\widehat{\alpha}_2(y_1) = 0$ und $\widehat{\alpha}_2(y_2) = \widehat{\alpha}_2(y_3) = 1$
      \item $\widehat{\alpha}_3(y_2) = 0$ und $\widehat{\alpha}_3(y_1) = \widehat{\alpha}_3(y_3) = 1$
      \item $\widehat{\alpha}_4(y_3) = 0$ und $\widehat{\alpha}_4(y_1) = \widehat{\alpha}_4(y_2) = 1$
    \end{itemize}
    Da $\widehat{\alpha}_i(x) = \alpha(x)$ für alle $x \in X$, erfüllen alle
    $\widehat{\alpha}_i$ die Klauseln $F_1,\dots,F_m$. Und da immer eine Variable
    aus $(y_1 \vee y_2 \vee y_3)$ auf 1 gesetzt wird, ist auch diese Klausel erfüllt.
    Wir haben also vier erfüllende Belegungen für $C$ und somit $C \in$ VIERFACH-SAT.

    ''$\Leftarrow$'': \\
    Sei $F \not\in$ SAT, d.h. es gibt keine Belegung, die alle Klauseln
    $F_1,F_2, \dots, F_m$ erfüllt. Es gibt also insbesondere keine Belegung
    für $C$, da $y_1, y_2, y_3$ nicht in $F$ vorkommen. Es gilt also
    $C \not\in$ VIERFACH-SAT.

    Somit ist VIERFACH-SAT NP-schwer und mit VIERFACH-SAT $\in$ NP schliessen
    wir, dass VIERFACH-SAT NP-Vollständig ist.
  \end{solution}}
\fi

\ifprintanswers
\newpage
\fi
    \part{}
    [\textbf{Aufgabe 4 -- Endterm 2017}] \\
    Wir nennen eine Klausel einer KNF-Formel \textit{monoton},
  wenn sie entweder keine negierten Variablen oder nur negierte Variablen enthält.
  Wir betrachten die Menge non-3-monotone-3SAT aller erfüllbaren KNF-Formeln,
  die aus Klauseln der Länge höchstens 3 bestehen und keine monotonen Klauseln der
  Länge genau 3 enthalten (Monotone Klauseln der Längen 2 und 1 sind somit erlaubt).
  Zeige, dass non-3-monotone-3SAT NP-vollständig ist.
\ifprintanswers
  \uplevel{\begin{solution}
    Es ist offensichtlich möglich eine Belegung in polynomieller Zeit zu prüfen.
    Es gilt also \nmSAT{} $\in$ NP.

    Wir zeigen nun 3SAT $\leq_p$ \nmSAT, was impliziert, dass \nmSAT{} auch NP-Schwer ist.

    Sei $F = F_1 \wedge F_2 \wedge \dots \wedge F_m$ eine Formel in 3KNF über die
    Variablen $X = \{x_1, \dots, x_n\}$.

    Wir konstruieren aus $F$ eine Eingabe $C$ für das \nmSAT{} Problem, so dass
    \begin{align*}
        F \in \text{3SAT} \iff C \in \text{\nmSAT}
    \end{align*}
    Dies tun wir wie folgt: Alle Klauseln welche eine Länge kürzer als 3 haben,
    bleiben unverändert. Eine Klausel $F_i = (l_{1,i} \vee l_{2,i} \vee l_{3,i})$
    ersetzen wir durch
    $(l_{1,i} \vee y_{1,i}) \wedge
      (\overline{y_{1,i}} \vee l_{2,i} \vee y_{2,i}) \wedge (l_{3,i} \vee \overline{y_{2,i}})$,
    wobei $y_{1,i},y_{2,i} \not\in X$ jeweils neue Variablen sind.
    Diese Konstruktion von $C$ ist offensichtlich in polynomieller Zeit möglich.

    Wir zeigen nun $F \in \text{3SAT} \iff C \in \text{\nmSAT}$:

    ''$\Rightarrow$'': \\
        Sei $F \in$ 3SAT, es gibt also eine erfüllende Belegung $\alpha$, die
        alle Klauseln $F_i$ erfüllt, d.h. mindestens ein Literal pro Klausel
        ist 1. Folgende Belegung $\beta$ erfüllt also $C$:
        $\beta(x) = \alpha(x)$ für alle $x \in X$ und
        \begin{align*}
          \beta(y_{1,i}) = \begin{cases}
              1 & \text{falls } l_{1,i} = 0 \\
              0 & \text{sonst}
          \end{cases} \hspace{3em}
          \beta(y_{2,i}) = \begin{cases}
              0 & \text{falls } l_{3,i} = 0 \\
              1 & \text{sonst}
            \end{cases}
        \end{align*}
        Alle Klauseln der Länge kleiner 3, werden trivialerweise erfüllt.
        Und mit $\beta(y_{1,i})$ wird garantiert, dass $(l_{1,i} \vee y_{1,i})$ erfüllt ist,
        und mit $\beta(y_{2,i})$ wird garantiert, dass
        $(l_{3,i} \vee \overline{y_{2,i}})$ immer erfüllt ist.
        Man bemerke, dass auch
        $(\overline{y_{1,i}} \vee l_{2,i} \vee y_{2,i})$ immer erfüllt ist.
        Somit $C \in$ \nmSAT.

    ''$\Leftarrow$'': \\
        Sei $F \not\in$ 3SAT, es gibt also keine erfüllende Belegung $\alpha$, d.h.
        es gibt eine Klausel $F_i$ welche nicht erfüllt ist. Hat die Klausel eine Länge
        kleiner 3, so ist der Fall trivial. Hat die Klausel die Länge drei, so sehen wir,
        dass es auch keine erfüllende Belegung für die Konstruierte Formel gibt. Da
        $\alpha(l_{1,i}) = \alpha(l_{2,i}) = \alpha(l_{3,i}) = 0$ muss gem. Konstruktion
        $\beta(y_{1,i}) = 1$ und $\beta(y_{2,i}) = 0$ gelten, somit ist also
        $(\overline{y_{1,i}} \vee l_{2,i} \vee y_{2,i})$ nicht erfüllt. Wir haben
        also $C \not\in$ \nmSAT.

  \end{solution}}
\fi
  \end{parts}

\end{questions}

\end{document}
