\documentclass[a4paper,ngerman,12pt]{exam}
\usepackage{babel}
\usepackage[utf8]{inputenc}
\usepackage[T1]{fontenc}
\usepackage{graphicx}
\usepackage{algpseudocode}
\usepackage{csquotes} % Anführungszeichen
\usepackage{paralist} % kompakte Aufzählungen
\usepackage{textcomp,tikz} %diverses
\usepackage{amsmath,amssymb,amstext,amsthm}
\usepackage{listings}
\usepackage{mathtools}
\usepackage{mdframed} % Boxen
\usepackage{geometry}
\usepackage{float}
\usepackage{tikz}
\usetikzlibrary{calc}
\usetikzlibrary{arrows, automata}

\geometry{a4paper, top=3cm, left=2.7cm, right=2.7cm}
\pagestyle{plain}
\renewcommand{\solutiontitle}{\noindent\textbf{Lösung:}\enspace}
\DeclarePairedDelimiter\ceil{\lceil}{\rceil}
\DeclarePairedDelimiter\floor{\lfloor}{\rfloor}
\renewcommand\L{\mathcal{L}}
\newcommand\ST{\Sigma_{\mathrm{T}}}
\newcommand\SN{\Sigma_{\mathrm{N}}}

%\printanswers

\begin{document}
\noindent Theoretische Informatik \hfill Gruppe 8 \\
\mbox{}\hfill Loris Reiff
\begin{center}
  \bfseries\Large
  Quiz 8 \ifprintanswers
  -- Lösungen
\fi
\end{center}

\begin{questions}
  \question
  Das Konzept der Grammatiken ist genau so stark wie das Konzept der Turingmaschinen
  \begin{checkboxes}
    \CorrectChoice Wahr
    \choice Falsch
  \end{checkboxes}

\question Entwerfe eine Grammatik für die Sprache der Palindrome, i.e. für
  \begin{align*}
    L = \{w \in \{a,b\}^* \mid w = w^{\text{R}}\}
  \end{align*}
  \vspace*{-2em}
\begin{solutionorbox}[8em]
  $G = (\SN, \ST, P, S)$ mit
  \begin{enumerate}[(i)]
    \item $\SN = \{S\}$,
    \item $\ST = \{a,b\}$,
    \item $P = \{S \to \lambda, S \to a, S \to b, S \to aSa, S \to bSb\}$
  \end{enumerate}
\end{solutionorbox}

\question Sei
  $G = (\{S,X\}, \{0,1\}, P, S)$ mit
  \begin{align*}
    P = \{S \to X00X, S \to X11X, X \to X1, X \to X0, X \to \lambda\}
  \end{align*}
  \begin{parts}
    \part
Bestimme die generierte Sprache.
    \uplevel{\begin{solutionorbox}[5em]
  $\{x \in \{0,1\}^* \mid x \text{ enthält 00 oder 11 als Teilwörter}\}$
    \end{solutionorbox}}
    \part Bestimme eine reguläre (Typ-3) Grammatik, welche äquivalent ist.
    Also die gleiche Sprache generiert.
    \uplevel{\begin{solutionorbox}[18em]
  $G = (\SN, \ST, P, S)$ mit
  \begin{enumerate}[(i)]
    \item $\SN = \{S, X\}$,
    \item $\ST = \{0,1\}$,
    \item
        $P = \{S \to 0S \mid 1S \mid 00X \mid 11X$ \\
        \hspace*{2.1em} $X \to 0X \mid 1X \mid \lambda \}$
  \end{enumerate}
    \end{solutionorbox}}
  \end{parts}


\end{questions}

\end{document}
